\chapter{绪~~论}
\fancyhead[CO]{\leftmark~~绪~~论}
\section{研究背景}
多时间尺度动力系统已成为当前非线性动力学研究领域热点之一。 这类系统广泛存在于自然界以及工程应用中,如小到微观的化学反应模型、激光模型以及神经元模型等;大到宏观的气候模型,生态系统以及天体力学等。 多时间尺度系统是指系统中不同的变量之间存不同量级的变化率\upcite{11zgy},通常在描述这类问题时,往往是通过构造具有快慢变量的微分方程组,因此被称为多时间尺度快慢系统。而由于多时间尺度快慢系统中变量之间通常存在不同的变化率,从而导致系统出现相应的奇异性,因此快慢系统是一类典型的奇异摄动系统\upcite{12jsy}。其所具有的复杂动力学行为的分析以及求解一直是研究的难点之一。目前大多数工作主要考虑的是具有两时间尺度的动力系统
\upcite{127lx,13ref_model_en,1101Broer2013Geometric},同时也存在针对更多时间尺度的研究\upcite{9yzq,15de2014three,128zsz}。



\section{研究现状}

\section{本文研究内容}

\newpage
\mbox{}
\newpage 