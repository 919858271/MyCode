%\setlength{\baselineskip}{20pt}
\linespread{1.5}
\bibliographystyle{setup/GB7714}
\setlength{\bibsep}{0.0ex}
%\numberwithin{equation}{chapter}
\renewcommand{\theequation}{\thechapter-\arabic{equation}}
\newdateformat{cdate}{\THEYEAR~年~\THEMONTH~月 }%中文日期显示格式
\newdateformat{edate}{\monthname[\THEMONTH],~\THEYEAR}%英文日期显示格式
%%%%%%%%%%%%%%%%%%%%%%%%%%%%%%%%%%%%%%%%%%%%%%%%%%%%%%%%%%%
% 重定义字体 字号命令
%%%%%%%%%%%%%%%%%%%%%%%%%%%%%%%%%%%%%%%%%%%%%%%%%%%%%%%%%%%
\newcommand{\song}{\CJKfamily{song}}    % 宋体   (Windows自带simsun.ttf)
\newcommand{\fs}{\CJKfamily{fs}}        % 仿宋体 (Windows自带simfs.ttf)
\newcommand{\kai}{\CJKfamily{kai}}      % 楷体   (Windows自带simkai.ttf)
\newcommand{\hei}{\CJKfamily{hei}}      % 黑体   (Windows自带simhei.ttf)
\newcommand{\li}{\CJKfamily{li}}        % 隶书   (Windows自带simli.ttf)
\newcommand{\yihao}{\fontsize{26pt}{26pt}\selectfont}       % 一号, 1.倍行距
\newcommand{\xiaoyi}{\fontsize{24pt}{24pt}\selectfont}      % 小一, 1.倍行距
\newcommand{\erhao}{\fontsize{22pt}{22pt}\selectfont}       % 二号, 1.倍行距
\newcommand{\xiaoer}{\fontsize{18pt}{18pt}\selectfont}      % 小二, 单倍行距
\newcommand{\sanhao}{\fontsize{16pt}{16pt}\selectfont}      % 三号, 1.倍行距
\newcommand{\xiaosan}{\fontsize{15pt}{15pt}\selectfont}     % 小三, 1.倍行距
\newcommand{\sihao}{\fontsize{14pt}{14pt}\selectfont}       % 四号, 1.0倍行距
\newcommand{\banxiaosi}{\fontsize{13pt}{13pt}\selectfont}   % 半小四, 1.0倍行距
\newcommand{\xiaosi}{\fontsize{12pt}{12pt}\selectfont}      % 小四, 1.倍行距
\newcommand{\dawuhao}{\fontsize{11.5pt}{11.5pt}\selectfont} % 大五号, 单倍行距
\newcommand{\wuhao}{\fontsize{10.5pt}{10.5pt}\selectfont}   % 五号, 单倍行距
\newcommand{\xiaowu}{\fontsize{9.5pt}{9.5pt}\selectfont}    % 五号, 单倍行距

\newcommand{\banbanxiaosi}{\fontsize{12pt}{12pt}\selectfont}% 半半小四, 1.0 倍行距
%%%%%%%%%%%%%%%%%%%%%%%%%%%%%%%%%%%%%%%%%%%%%%%%%%%%%%%%%%%
% 页眉样式
%%%%%%%%%%%%%%%%%%%%%%%%%%%%%%%%%%%%%%%%%%%%%%%%%%%%%%%%%%%
\fancypagestyle{plain}{
  \pagestyle{fancy}
}
\newcommand{\makeheadrule}{%
\makebox[-3pt][l]{\rule[.76\baselineskip]{\headwidth}{0.4pt}}
\rule[0.85\baselineskip]{\headwidth}{2.35pt}\vskip-.8\baselineskip}
\renewcommand{\headrule}{%
    {\if@fancyplain\let\headrulewidth\plainheadrulewidth\fi
     \makeheadrule}}
\pagestyle{fancyplain}
\fancyhf{}
\renewcommand{\chaptermark}[1]{\markboth{\xiaowu\song{第\chinese{chapter} 章}}{}}
\fancyhead[CO]{ }
\fancyhead[LO]{ }
\fancyhead[CE]{\xiaowu\song{西北工业大学硕士学位论文}}
\fancyfoot[C,C]{\xiaowu{\thepage}}

%%%%%%%%%%%%%%%%%%%%%%%%%%%%%%%%%%%%%%%%%%%%%%%%%%%%%%%%%%%
% 目录样式
%%%%%%%%%%%%%%%%%%%%%%%%%%%%%%%%%%%%%%%%%%%%%%%%%%%%%%%%%%%
\renewcommand{\contentsname}{{\sanhao\hei目~~录}}
\CTEXsetup[name={第,章},number={\chinese{chapter}}]{chapter}
\CTEXsetup[number={\thesection}]{section}
\CTEXsetup[number={\thesubsection}]{subsection}
%tocloft和ctexcap标题重叠现象
\makeatletter
\renewcommand{\numberline}[1]{%
\settowidth\@tempdimb{#1\hspace{0.5em}}%
\ifdim\@tempdima<\@tempdimb%
  \@tempdima=\@tempdimb%
\fi%
\hb@xt@\@tempdima{\@cftbsnum #1\@cftasnum\hfil}\@cftasnumb}
\makeatother
%tocloft和ctexcap标题重叠现象
\titlecontents{chapter}[0mm]
%{\vspace{.2\baselineskip}\bf}
{\vspace{.01\baselineskip}}
{{\thecontentslabel}\hspace{0.1em}} {}
{\dotfill\contentspage[{\makebox[0pt][r]{\thecontentspage}}]}
[\vspace{.01\baselineskip}]
\titlecontents{section}[18pt]
{\vspace{.01\baselineskip}}
{{\thecontentslabel}\hspace{0.6em}} {}
{\dotfill\contentspage[{\makebox[0pt][r]{\thecontentspage}}]}
[\vspace{.01\baselineskip}]
\titlecontents{subsection}[40pt]
{\vspace{.01\baselineskip}}
{{\thecontentslabel}\hspace{0.6em}}{}
{\dotfill\contentspage[{\makebox[0pt][r]{\thecontentspage}}]}
[\vspace{.01\baselineskip}]
%%%%%%%%%%%%%%%%%%%%%%%%%%%%%%%%%%%%%%%%%%%%%%%%%%%%%%%%%%%
% 标题样式
%%%%%%%%%%%%%%%%%%%%%%%%%%%%%%%%%%%%%%%%%%%%%%%%%%%%%%%%%%%
\titleformat{\chapter}[block]{\centering\sanhao\hei}{第\chaptername章}{1em}{}
\titleformat{\section}[hang]{\hei\sihao}{\sihao\hei\thesection}{0.5em}{}
\titleformat{\subsection}[hang]{\hei\xiaosi}{\xiaosi\hei\thesubsection}{0.5em}{}
\titleformat{\subsubsection}[hang]{\hei\xiaosi}{\xiaosi\hei\thesubsubsection}{0.5em}{}
\titlespacing{\chapter}{0pt}{0mm}{3mm}
\titlespacing{\section}{0pt}{1.5mm}{1.5mm}
\titlespacing{\subsection}{0pt}{1.5mm}{1.5mm}
%\titlespacing{\subsubsection}{0pt}{1.5mm}{1.5mm}
%%%%%%%%%%%%%%%%%%%%%%%%%%%%%%%%%%%%%%%%%%%%%%%%%%%%%%%%%%%
% 公式,图表,定理
%%%%%%%%%%%%%%%%%%%%%%%%%%%%%%%%%%%%%%%%%%%%%%%%%%%%%%%%%%%
\renewcommand\refname{参考文献}
\renewcommand{\figurename}{图}
\renewcommand{\thefigure}{\thechapter-\arabic{figure}}
\renewcommand{\tablename}{表}
\renewcommand{\thetable}{\thechapter-\arabic{table}}
\newtheorem{Proof}{证明}[chapter]
\newtheorem{theorem}{定理}[chapter]
\newtheorem{lemma}{引理}[chapter]
\newtheorem{remark}{注记}[chapter]
\newtheorem{definition}{定义}[chapter]
\newtheorem{proposition}{命题}[chapter]
\def\propositionautorefname{命题}
\def\theoremautorefname{定理}
\def\lemmaautorefname{引理}
\def\definitionautorefname{定义}
\def\remarkautorefname{注记}
\def\figureautorefname{图}
\def\subfigureautorefname{图}
\def\tableautorefname{表}
\crefname{equation}{方程}{方程}
\crefname{table}{表}{表}
\crefname{figure}{图}{图}
\captionsetup[figure]{labelsep=space}
\captionsetup[table]{labelsep=space}
\newcommand{\upcite}[1]{\textsuperscript{{\cite{#1}}}}


%表格样式
\newlength\savedwidth
\newcommand\whline{\noalign{\global\savedwidth\arrayrulewidth
                            \global\arrayrulewidth 1.2pt}
                   \hline
                   \noalign{\global\arrayrulewidth\savedwidth}}
\newlength\savewidth
\newcommand\shline{\noalign{\global\savewidth\arrayrulewidth
                            \global\arrayrulewidth 1.2pt}
                   \hline
                   \noalign{\global\arrayrulewidth\savewidth}}
%===================== 自动断字设置 =====================%
\tolerance=1
\emergencystretch=\maxdimen
\hyphenpenalty=100000
\hbadness=100000